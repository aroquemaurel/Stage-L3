\chapter{Bilans}
	\section{Bilan pour Continental}
	Mon stage va se prolonger jusqu'au 11 Juillet, ainsi le bilan du projet au moment de la rédaction de ce rapport n'est que partiel. 

	Mon travail dans l'équipe a été bénéfique, durant ce stage, j'ai aidé à la conception de la plateforme, notamment au niveau des Alias, des Devices, de l'analyse des traces et du parser,	j'ai aussi apporté un regard neuf sur le travail déjà effectué.

	Mon arrivée à apporté une personne de plus à la conception, or il est plus facile de concevoir un système facilement maintenable à 3 personnes. Au vu de nos
	approches différentes, concevoir un système nous satisfaisant tous les trois devrait être compréhensible pour le plus grand nombre.
	
	J'ai développé une bonne partie de la génération, j'ai ainsi pu permettre à l'équipe de gagner du temps et de
	tenir au mieux les délais. À l'heure actuelle, la partie sur les \textit{Expected Behavior} n'est pas terminée, cependant la conception étant faite cela devrait aller
	assez rapidement.
	
	Je n'ai malheuresement pu tester mon travail qu'en local, afin de pouvoir tester sur table, il faut terminer un prototype de TestManager. Celui-ci étant bien
	avancé et la création des traces étant presque finie, une fois l'analyse des traces fonctionnelles, nous pourrons effectuer nos premiers tests.

	Le bilan pour l'entreprise semble donc être très positif, ce qui me motive pour la suite, notamment le mois de stage restant !

	\section{Bilan personnel}
	Cette expérience en entreprise m'a beaucoup apporté, tout d'abord d'un point de vue technique, j'ai acquis de l'expérience en conception logicielle, grâce
	à toutes nos réunions où nous réfléchissions à la meilleure approche possible. De plus lors de problèmes, les propositions des autres m'ont permis d'avoir
	une autre vision du problème et une autre manière de le résoudre !

	Mais j'ai aussi acquis des connaissances humaines avec notamment le travail en équipe, communiquer sur nos avancements, et être capable de synthétiser ses
	propositions ou de réussir à poser un problème rapidement tout en se faisant comprendre.

	Ce stage m'a permis de découvrir le monde d'une grande multinationale, jusqu'à maintenant je ne connaissais que les PME et n'étais pas sûr de pouvoir
	travailler dans une grande entreprise.

	Un bilan très positif, qui m'a réconforté dans mon projet professionnel : le développement logiciel et la conception sont vraiment les domaines de
	l'informatique qui m'intéressent le plus, je souhaite poursuivre vers un master Développement Logiciel.
