\chapter*{Introduction}
	\addcontentsline{toc}{chapter}{Introduction} 
	Dans le cadre de ma formation en 3ème année de licence à l'université Toulouse III -- Paul Sabatier, j'avais le choix entre effectuer un TER\footnote{Travail Etude Recherche} ou un stage.

	J'ai fait le choix d'un stage, car je suis bien plus attiré par le monde de l'entreprise que celui de la recherche, d'autant plus que j'étais à la recherche d'un stage de 3 mois me permettant de prolonger ce travail et d'avoir un sujet qui me paraissait plus intéressant.

	J'ai eu la chance d'avoir une opportunité de stage dans l'entreprise Continental Automotive, afin d'effectuer du développement logiciel. J'ai été rapidement
	séduit par le sujet : le développement d'une plateforme de tests de logiciels embarqués. En effet, lors d'un précédent stage, j'ai travaillé dans le web, je voulais effectuer un stage dans le logiciel, cette branche du développement logiciel m'intéressant plus. Or les tests logiciels sont extrêmement importants, particulièrement dans le monde de l'automobile où une simple erreur peut être fatale.\\
	Souhaitant travailler dans une grande entreprise afin de comparer le travail vis-à-vis d'une PME, Continental était donc le choix parfait pour mon stage.

	Ainsi, le sujet m'a été présent plus en détails par mail : l'équipe Vérification et Validation doit automatiser les tests d'un plugin réalisant la plus grande partie des stratégies applicatives permettant de contrôler le moteur du véhicule cible : l'utilisateur devra donner en entrée un fichier décrivant les cas de tests, et son poste accédera à distance à des tables de tests afin d'exécuter ceux-ci.\\
	L'automatisation de ces tests permettrai à l'équipe en charge de ceux-ci de gagner beaucoup de temps d'une part, et d'autres parts, limiterai au maximum les risques d'erreurs humaines.

	Dans ce rapport, nous allons voir en quoi le développement de cet outil est nécessaire à l'équipe en charge des tests de ce plugin. Dans une première partie, nous présenterons l'entreprise Continental, et l'équipe Vérification et Validation plus en détails, ensuite nous aborderons le problème que pose les tests de ce plugin actuellement et nous verrons ensuite la solution qui est en cours de développement, et comment j'ai contribué à ce projet.
