\chapter*{Introduction}
	\addcontentsline{toc}{chapter}{Introduction} 
Dans le cadre de ma formation en troisième année de licence à l'université Toulouse III – Paul Sabatier, j'ai eu le choix entre effectuer un TER\footnote{Travail Etude Recherche} ou un stage. 

J'ai choisi la seconde option car je me sens plus attiré par le monde de l'entreprise que par la recherche. J'ai eu la chance d'avoir une opportunité de stage de trois mois dans l'entreprise Continental Automotive au sein de l'équipe Vérification et Validation pour un projet de développement d'une plateforme de tests de logiciels embarqués.

Souhaitant intégrer le Master << Développement logiciel >>, le sujet même du stage était particulièrement intéressant. En outre, après une expérience de quatre mois dans une PME au cours de mon cursus à l'IUT 'A' Paul Sabatier, il me semblait pertinent de travailler au sein d'une grande entreprise à caractère international pour mieux définir mon projet professionnel.

Après une première phase de contact, le projet m'a été présenté plus en détails par mail. Un fournisseur distribue un plugin à Continental : un bout de code sous
forme d'objet, que l'entreprise doit être intégrer dans le logiciel applicatif d'un calculateur de contrôle moteur. La mission de l'équipe Vérification \&
Validation est de tester la bonne intégration de ce plugin dans l'applicatif du calculateur, pour cela une plateforme qui permettra des tests automatisés est en
cours de développement.

Dans ce rapport nous verrons en quoi le développement de cet outil est nécessaire à l'équipe en charge des tests de ce plugin. Dans une première partie nous présenterons l'entreprise Continental et plus particulièrement l'équipe Vérification et Validation. Nous aborderons ensuite le problème que posent actuellement les tests de ce plugin, avant de présenter la solution qui est en cours de développement et comment j'ai contribué à ce projet.
