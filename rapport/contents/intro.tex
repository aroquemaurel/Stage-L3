\chapter*{Introduction}
	\addcontentsline{toc}{chapter}{Introduction} 
	Dans ce rapport, nous allons voir en quoi le développement de cet outil est nécessaire à l'équipe en charge des tests de ce plugin. Dans une première partie, nous présenterons l'entreprise Continental, et l'équipe Vérification et Validation plus en détails, ensuite nous aborderons le problème que pose les tests de ce plugin actuellement et nous verrons ensuite la solution qui est en cours de développement, et comment j'ai contribué à ce projet.

Dans le cadre de ma formation en troisième année de licence à l'université Toulouse III – Paul Sabatier, j'ai eu le choix entre effectuer un TER\footnote{Travail Etude Recherche} ou un stage. 

J'ai choisi la seconde option car je me sens plus attiré par le monde de l'entreprise que par la recherche. J'ai eu la chance d'avoir une opportunité de stage de trois mois dans l'entreprise Continental Automotive au sein de l'équipe Vérification et Validation pour un projet de développement d'une plateforme de tests de logiciels embarqués.

Souhaitant intégrer le Master << Développement logiciel >>, le sujet même du stage était particulièrement intéressant. En outre, après une expérience de quatre mois dans une PME au cours de mon cursus à l'IUT 'A' Paul Sabatier, il me semblait pertinent de travailler au sein d'une grande entreprise à caractère international pour mieux définir mon projet professionnel.

Après une première phase de contact et avant signature de la convention, le projet m'a été présenté plus en détails par mail. L'équipe Vérification et Validation doit automatiser les tests d'un plugin réalisant la plus grande partie des stratégies applicatives permettant de contrôler le moteur du véhicule cible ; l'utilisateur devra donner en entrée un fichier décrivant les cas de tests puis accédera à distance à des tables de tests afin de les exécuter. L'objectif de l'automatisation est double puisqu'il permettrait un gain de temps aux équipes et limiterait au maximum les risques d'erreurs humaines.

Dans ce rapport nous verrons en quoi le développement de cet outil est nécessaire à l'équipe en charge des tests de ce plugin. Dans une première partie nous présenterons l'entreprise Continental et plus particulièrement l'équipe Vérification et Validation. Nous aborderons ensuite le problème que posent actuellement les tests de ce plugin, avant de présenter la solution qui est en cours de développement et comment j'ai contribué à ce projet.
