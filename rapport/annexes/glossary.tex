\chapter{Acronymes et Glossaire}
\begin{description}
\item[Antlr] Another Tool for Language Recognition, outil permettant de faciliter l'interprétation d'une chaîne de caractère, celui-ci prend en entrée une
	grammaire, et génère un arbre syntaxique dans plusieurs langages.
\item[Device] Les différents équipements dont pourrait avoir besoin l'utilisateur : Hil, Debugger, \ldots 
\item[ECU] Electronic Control Unit, calculateur du contrôle moteur
\item[Grammaire] Formalisme permettant de définir une syntaxe clair et non ambigüe.
\item[HIL] Hardware in the loop, permet de simuler un environnement véhicule autour du calculateur du contrôleur moteur : celui-ci réagira comme s'il était embarqué dans une voiture.
\item[JAR] Java ARchive est un fichier ZIP utilisé pour distribuer un ensemble de classes Java.
\item[Java] Langage de programmation orienté Objet soutenu par Oracle. Les exécutables Java fonctionnent sur une machine virtuelle Java et permettent d'avoir un
	code qui soit portable peut importe l'hôte.
\item[JSON] JavaScript Object Notation est un format de données textuelles, générique, dérivé de la notation du langage JavaScript, il permet de représenter de
	l'information structurée.
\item[JVM] Java Virtual Machine
\item[Logiciel de versionnement] Logiciel, tel que \textit{Git}, permettant de maintenir facilement toutes les versions d'un logiciel, mais aussi facilitant le
	travail collaboratif.
\item[Parsing] Processus d'analyser de chaîne de caractère, en supposant que la chaîne respecte un certain formalisme. 
\item[Apache Thrift] Langage de définition d'interface conçu pour la création et la définition de services pour de nombreux langages. Il est ainsi possible de
	faire communiquer deux problèmes dans deux langages différents : Python et Java dans notre cas.
\item[Trace32] Debugger, permet de debugger un programme embarqué, ceci en permettant de lire la mémoire, mettant des points d'arrêts, \ldots
\item[UML] Unified Modeling Language est un langage de modélisation graphique. Il est utilisé en développement logiciel et en conception orienté Objet afin de
	représenter facilement un problème et sa solution.
\item[XML] Extensible Markup Language est un langage de balisage générique permettant de stocker des données textuelles sous forme d'information structurée.
\end{description}
% HIL
% Trace32
% UML
% Logiciel de versionnement
% ECU
% JSON
% Thrift
% Grammaire
% Antlr
% Parser

